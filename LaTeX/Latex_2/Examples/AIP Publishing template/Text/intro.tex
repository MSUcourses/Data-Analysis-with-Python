Current~\cite{girshick2014rich} state-of-the-art object detectors are based on
a two-stage, proposal-driven~\cite{lin2017focal} mechanism. As popularized
in the R-CNN framework, the first stage generates a
sparse set of candidate object locations and the second stage
classifies each candidate location as one of the foreground
classes or as background using a convolutional neural network. Through a sequence of advances , this
two-stage framework consistently achieves top accuracy on
the challenging COCO benchmark.
Despite the success of two-stage detectors, a natural
question to ask is: could a simple one-stage detector achieve
similar accuracy? One stage detectors are applied over a
regular, dense sampling of object locations, scales, and aspect ratios. Recent work on one-stage detectors, such as
YOLO and SSD, demonstrates promising
results, yielding faster detectors with accuracy within 10-
40\% relative to state-of-the-art two-stage methods.

Current~\cite{girshick2014rich} state-of-the-art object detectors are based on
a two-stage, proposal-driven~\cite{lin2017focal} mechanism. As popularized
in the R-CNN framework, the first stage generates a
sparse set of candidate object locations and the second stage
classifies each candidate location as one of the foreground
classes or as background using a convolutional neural network. Through a sequence of advances , this
two-stage framework consistently achieves top accuracy on
the challenging COCO benchmark.
Despite the success of two-stage detectors, a natural
question to ask is: could a simple one-stage detector achieve
similar accuracy? One stage detectors are applied over a
regular, dense sampling of object locations, scales, and aspect ratios. Recent work on one-stage detectors, such as
YOLO and SSD, demonstrates promising
results, yielding faster detectors with accuracy within 10-
40\% relative to state-of-the-art two-stage methods.